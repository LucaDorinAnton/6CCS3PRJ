\chapter{Next Steps and Conclusions}

\section{Next Steps}
While a significant amount of work has been done towards making the 16-bit breadboard
computer a polished project, there are many goals which can still be achieved if the
project were to be continued.
\begin{itemize}
  \item \emph{I/O Troubleshooting:} The I/O module is not fully functional. A first step
  towards improving the 16-bit breadboard computer would be to ensure that this module
  works as specified as well.
  \item \emph{More Complex Instruction Set: } currently, the instruction set defined for the 16-bit
  breadboard computer is mostly composed of instructions which execute very simple tasks, like moving
  a value between registers or adding up two numbers. It might be interesting to see how the
  functions of the computer, as well as the design of the assembler and compiler would change given
  a minimal instruction set and an instruction set designed to be as complex as possible.
  \item \emph{Compiler optimizations: } The Compiler produced in this project  achieves the goal
  of compiling \emph{WHILE} language sources down to assembly, but it doesn't apply any compiler
  optimizations. For example, the multiply subroutine, which currently essentially just adds the
  multipliand with itself as many times as the multiplier, could be replaced with a shift-add
  subroutine.
  \item \emph{Added functionality } currently, the computer handles only unsigned integers. It would be
  interesting to see if it is possible to extend to signed integers, characters, strings and arrays
  without adding any more hardware.
  \item \emph{More complex programs: } Another potential goal would be to test the 16-bit breadboard
  computer with more complex programs. A good starting point might be sorting algorithms.
  \item \emph{LLVM integration: } The LLVM suite of programs has at its core the \emph{LLVM IR}
  \cite{llvm}, or \emph{Intermediate representation} This low-level language is designed to sit between
  hardware and  programming languages and allows programming language designers to compile their
  language down to the IR and then run on any hardware which implements a compiler from the IR to their
  particular instruction set. It would be interesting to see if it possible to write a compiler from
  a subset of the IR and then use this abstract language interface to run programs written in modern
  high-level languages on the 16-bit breadboard computer.
\end{itemize}


\section{Conclusion}
The 16-bit breadboard computer architecture project ended up being an outstanding learning experience
for learning about computer architecture, processor design, hardware design and low level software
design and implementation. The end product is something that can complete many of the tasks that
modern computers complete, while still being simple and understandable down to a very low level of
abstraction. This report can serve as a guide for anyone who might want to attempt a similar
undertaking or just focus on learning about computer architecture. While offering few practical
advantages, such a computer can be of significant use in modern life, not just as a teaching medium,
but also as a way for a software engineer to become more connected with the hardware he uses in his
day to day activities and to have a better grasp of the capabilities and limitations of the tools
at his disposal.
