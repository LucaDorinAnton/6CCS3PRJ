\chapter{Software Design}

\section{Microcode format and programmer}

\subsection{Design Choices}

\paragraph{Language choice}
Given the fact that the microcode programmer will be tasked with processing
both text and binary files, \emph{Python} \cite{py} comes up as an obvious choice
for an implementation language. Python makes file manipulation and text
processing enjoyable instead of frustrating, as in many other languages, because it
requires essentially no boilerplate to achieve those goals. Moreover, Python is an
implements object-oriented philosophy, which will facilitate the creation of an internal
model for the instructions which are to be programmed. The syntax is also light weight,
concise and understandable, which makes \emph{Python} a good choice for a scripting
language as well. This is important because the microcode programmer will be a command-line
script.

\paragraph{Microcode Instruction Format}
Given the specification for a simple to learn and used format for declaring
microcode instructions, two possiblities come to mind:

\paragraph{Language specific} One approach would be to create classes for the
microcode instructions in some object-oriented programming language. This way,
a user who is familiar with the language could easily create multiple objects which
are instantiations of that particular class. After the user has created the class
instances, he could run a script to generate microcode instructions from them.
Since the implementation language chosen is \emph{Python}, Instructions would be
represented as Python objects.

\paragraph{Language agnostic} Another approach would be to make use of a format
which doesn't depend on a specific programming language. The main advantage of this
approach would that someone who isn't familiar with the implementation language
could use the tool efectivley. This would also mean that the programmer would
have to contain a parser for this format, separate from the parser of the implementation
language. A good choice of format for such an approach might be \emph{YAML} \cite{yaml}.
\emph{YAML Ain't Markup Language} (recursive definition), is a data serialization language
designed around the main goal of being human-readable \cite{yaml}. As such, it poses itself
as a great choice for the microcode definiton format, as it is so simple that one
does not have to be familiar with it to quickly undestand it and use it. Both approaches
towards formatting microcode instructions should be implemented.

\subsection{Programmer design}
Given a file in either of the two formats discussed in the previous section, the programmer
should follow the following broad steps
\begin{enumerate}
  \item If provided a Python source file, parse it and extract the instruction objects
  \item If provided a YAML file, parse it and then convert the YAML objects to Python instruction
  objects
  \item Generate 4 bytearrays containing zero bytes. The bytearrays should be of the size of the
  memory arrays of the control logic EEPROMs
  \item For each instruction, for each flag environment, for each step:
  \item Concatenate the binary formatting of the opcode, flag environment and time step
  \item Interpret the resulting string as a binary number and convert it to an integer
  \item Use the resulting integer as a pointer to write
  the control signal bytes of that timestep to the right place in the bytearrays
  \item If no control signal sequence is provided for a certain flag environment, use the default
  one
  \item Write the resulting bytearrays to files
\end{enumerate}


\section{Assembler}

\subsection{Design choices} \label{asm-des}

\paragraph{Language} For the assembler for the 16-bit computer the choice was made to
use a different language then the one used for the microcode programmer, namely \emph{Scala}
\cite{scala}. \emph{Scala} is a modern multi-faceted multi-paradigm all-purpose built on top
of the Java ecosystem. \emph{Scala} is large-scale widley-used and fully-featured to such an
extent that it would be infeasable to cover all of its features in this report. However, noting
the most important language features will shed light onto why it poses itself as a great tool
for implementing assemblers and compilers:

\paragraph{Functional programming}
Scala has all the functionality needed to implement programs and scripts following a functional
paradigm. This feature lends itself to the advantage in the task of implementing an assembler
and compiler because of the natural tendency to solve assembling and compiling tasks using
recursive functions. Functional programming encourages and simplifies the creation of recursive
functions.

\paragraph{Java Ecosystem}
Since \emph{Scala} is built on top of \emph{Java} \cite{java}, the entire \emph{Java Ecosystem}
is a subset of \emph{Scala}. As such, \emph{Scala} doesn't confine the developer to functional programming, allowing great freedom in mixing and matching styles and paradigms.

\paragraph{Robust yet Concise Typesystem}
\emph{Scala} comes with all the benefits of the \emph{Java} typesystem, including stron typing,
type checks, object oriented approach, subclasses, abstract classes and many other, while at the
same time not having the same downsides. The main advantage of the \emph{Scala} typesystem is the
fact that it requires significantly less boilerplate than its \emph{Java} counterpart. This
feature is useful given the inherent need of compilers and assemblers generally to create many
internal data structures to faciliate the transformation of source code from one format to the
next.


\paragraph{Dynamic and flexible Syntax}
The \emph{Scala} syntax is concise and understandable, similar to Python in many ways. Unlike
Python syntax, it has some key advantages. It doesn't rely on indentation to determine scope,
which can be a source of frustration allthough it makes code much more readable and it gives
the user the option to extend and augment the syntax using implicit definitions. This makes
the life of the assembler and compiler designer easier by allowing him to write shortcuts in
places were repetitive tasks are undertaken.

\paragraph{YAML Parser}
Since the assembler will have to be aware of any changes in instruction set defined by the user
for the 16-bit breadboard computer, the need for a YAML parser arrises. This parser will be used
to process the same list of instructions used by the microcode programmer and then collect a list
of mnemonics. Using this list, the parser can determin if the assembly source file conforms to
that particular instruction set.

\subsection{Assembler Design}
The Assembler will be built around a two-module design. A \emph{Lexer} and a \emph{Code
generator}.

\paragraph{Lexer}
The task of the \emph{Lexer} is to transform the contents of an assembly file, essentially a
simple string, into a list of tokens. Then, the \emph{Lexer} can pass on the token list to the
\emph{Code generator}. The \emph{Lexer} uses \emph{regular expressions} to read the input string
and break it down into tokens. When a piece of the input string matches a particular token
pattern, an instance of that token is created with that specific piece and then added to the
list. Then the \emph{Lexer} continues to process the remaining input string. Besides this, the
\emph{Lexer} also has the task of eliminating whitespace and comments from the token list,
as these play no role in code generation. The lexer contains a custom-built regular expression
engine built around the concept of \emph{regular expression derivates}. This alternate way
to think about and implement regular expressions is taught by \emph{Dr. Christian Urban} in the
\textbf{6CCS3CFL} course\cite{6ccs3cfl}. The use of derivative regular expressions significantly
improves performance in certain edge cases. When a match is found, an injection function is used
to rebuild a ``value'' from the derivative which points towards which token has matched.

\paragraph{Code generator}
The \emph{code generator} receives the list of tokens from the \emph{Lexer} and then does further
processing on it. Since assembly doesn't have a complex syntax or structure, there is no need
for a parses. The assembler completes the following steps to fullfil its task:
\begin{enumerate}
  \item Alias extraction
  \item Subroutine ordering
  \item Subroutine separation
  \item Reference solving
  \item Instruction decoding
\end{enumerate}
Once these steps are completed, the code generator is able to generate a bytearray and write a
binary file to disk.


\section{Compiler}
\subsection{Design Choices}
\paragraph{Language}
Since the Compiler and the Assembler share a significant amount of structure and code, the choice was made to also
use \emph{Scala} for the compiler. The reasons for this choice have been stated in the Assembler design choices
section \ref{asm-des}.
\paragraph{Structure}
The compiler has a similar structure to the assembler. It also has a \emph{Lexer} and a \emph{Code Generator}.
The main difference is the presence of a third module, namely the \emph{Parser}. The \emph{Parser} sits between
the \emph{Lexer} and the \emph{Code Generator} and is tasked with generating an \emph{Abstract Syntax Tree} from
the list of tokens.
\subsection{Compiler Design}
\paragraph{Lexer} The compiler \emph{Lexer} follows the same design pattern as the one used for the assembler.
The main difference between the two is that they use different regular expressions to parse the input string.

\paragraph{Parser} The \emph{Parser} has the task of taking the token list, which has no structure except for the
total ordering imposed onto it and then turn it into a \emph{Abstract Syntax Tree}, or \emph{AST} for short. An
\emph{AST} encapsulates the internal structure of a program, like for example nested loops, if-else branches and the
order of operations in an arithmetic expression. The \emph{Parser} of the compiler is based on a parsing philosophy
called ``parser combinators'', which is also taught by \emph{Dr. Christian Urban} in the \textbf{6CCS3CFL} module
\cite{6ccs3cfl}. Parser combinators work by defining simple as objects which try to parse as much of the input tokens
as possible and then return the amount they could parse, as well as the remaining part of the input which they
couldn't process. By taking multiple simple parsers and \emph{combining} them (hence the name), a more complex parser
can be created. Their nature is to return a set of pairs of processed and unprocessesed input. This way,
the next parser in line can try to parse based on all the different ways the previous parsers parsed.
When the infrastructure has been set up, writing parser combinators for a language can be as simple as declaring
the grammar of that particular language. Parser combinators make use of \emph{lazy evaluation}
which is a mode in which arguments are only evaluated at the last possible moment in time.

\paragraph{Code generator}
The code generation stage for the Compiler is simplified, since most of the processing has been completed by the
\emph{Parser}. In this case, the \emph{Code generator} has to traverse an abstract syntax tree recursivley and then
map its elements to chunks of code from the target language, in this case 16-bit breadboard computer assembly.


\section{Programming and I/O Arduino Drivers}
\subsection{Design choices}

\paragraph{Language} For the design of the programming and I/O Arduino drivers, \emph{Python} was chosen for the
script on the computer side and \emph{C} was chosen for the Arduino script. \emph{Python} is a great scripting
language and it handles file I/O, both binary and text, in an elegant manner. So it presented itself as a good
choice for the scripts running on the computer. \emph{C} was chosen for the Arduino scripts because it is the only
language supported by the Arduino Integrated Development Environment \cite{arduino_ide}.

\subsection{Script Design}
The scripts for running the Arduinos do not offer much in term of design depth, since they are relativley short
and simple. What is notable is the need to use interrupts to interface with the breadboard computer and the use of
the serial port to communicate to the computer. The Arduinos also make extensive use of bitshifting and bit
extraction to convert data from the internal datatypes to voltages on multiple pins and pin voltages to datatypes.
