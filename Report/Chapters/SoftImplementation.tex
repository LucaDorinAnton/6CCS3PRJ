\chapter{Software Implementation}
This chapter details the process through which the software previously designed
was implemented. Additionally, tools and libraries used were discussed.


\section{Implementation Process}
The different software tool were developed in sequence. This was because
each tool was dependent on the previous to work. After each short development
interation. Where possible, the software was tested against either the phyisical implementation
of the 16-bit breadboard computer or against test files.

\section{Libraries used}
Besides the standard libraries of both \emph{Python} and \emph{Scala}, as well as
the standard \emph{C++} functions provied by the \emph{Arduino IDE}, the following libraries
were also used to simplify and expedite the implementation process:
\paragraph{PyYAML} \cite{pyyaml} PyYAML provides a YAML parser for Python. In this case it was
used on the microcoed generator to parse instructions in YAML format. PyYAML converts a YAML
file into a python object consisting of lists, dictionaries and primitives.

\paragraph{moultingYAML} \cite{moultingyaml} MoultingYAML is a YAML parser for Scala. It works
more subtly than PyYAML, as it allows the user to specify a \emph{Protocol} with which to
parse a YAML file directly into a user-defined custom Scala object. It was used in the
assembler to parse instruction set files.

\paragraph{Scala Binary Files pattern} \cite{scala-bin} This gist was used as an inspiration
source for using Scala to write binary files to disk.

\section{Source-code basis}
The basis of the source code for both the assembler and the compiler is the fouth coursework
of the \textbf{6CCS3CFL} course taught by \emph{Dr. Christian Urban} \cite{6ccs3cfl}.

\section{Tools used}

\paragraph{Build Tools} Given the simplicity of the scripts, no build tools were used for the
implementation of the microcode programmer and the two arduino drivers. For the implementation
of the assembler and compiler, \emph{Scala Build Tools} (sbt) was used \cite{sbt}. Sbt handled
dependecies, installation of scala, the interactive interpreter and the running of the script.
