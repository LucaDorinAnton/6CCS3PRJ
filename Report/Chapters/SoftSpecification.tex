\chapter{Software Specification}
To transform the 16-bit breadoard computer from an exercise in logic design
into an advanced and rich learning meadium, a suite of feature-rich software
tools is needed.

\section{Main functionailty to be implemented}
The software for the breadboard computer should implement the following pieces
of high-level pieces of functionailty to be considered successfull:
\begin{itemize}
  \item \emph{Microcode definition and re-programming: } The computer needs the
  control logic EEPROM's \ref{control-logic} to be programmed with microcode. There
  should also be an accessible way to define and modify microcode.
  \item \emph{Assembler: } Given a set of instructions defined through microcode,
  there should be a way to write assembly instead of plain binaries and then assemble
  the code produced down to binaries
  \item \emph{Compiler: } Going up one more level of abstraction, one should be able
  to write a program in the WHILE programming language and then compile it down to
  either assembly or binaries directly.
  \item \emph{Programming arduino drivers: } Currently, the computer is programmed
  by setting the \emph{PROG} switch and then manually flipping dip switches on
  the word selector \ref{word-select} and the address selector \ref{addr-select}. This
  process is tedious, prone to mistakes and it can take a relativley long period of time
  to program a small number of instructions. Since an Arduino is connected to the word
  and address selectors, there should be a way to have th arduino program a binary produced
  by the assembler.
  \item \emph{I/O Arduino drivers: } While the I/O \ref{io} module has been built to its
  specification, it relies on the Arduino handling its read and write request. To achieve this,
  some driver software needs to be created.
\end{itemize}
The specification for each piece of software will be detailed in the following sections.

\section{Microcode Definition and Programming}
\subsection{Microcode format}
Currently, the only way to define instructions for the 16-bit breadboard computer is to write
binaries by hand, and then program them to the control logic EEPROMs. This is tedious and
prone to errors. There should exist a format through which instructions for the 16-bit breadboard
computer can be designed and stored. This format should be simple to read, understand, write and
change. The format should be high-level enough, so that the instruction designer doesn't have to
think about setting individual bits, while still being able to thik about which control signals
are set at which time step during which flag conditions. The format should specify the following
pieces of information regarding an instruction:
\begin{enumerate}
  \item Mnemonic (three-letter code)
  \item Opcode (6-bit number)
  \item Description (String)
  \item Whether the instruction requires an operand or not (boolean)
\end{enumerate}
Following this, each instruction should have associated to it at least one execution path.
This path should be dependant on the flag environment, but there should be at least one path
to serve as a default in all flag environments conditions which are not specified. An execution
path consists of a list of steps, each step detailing which control signals should be activated
at that particular time step.

 \subsection{Parsing and code generation}
 After a file declaring instructions conforming to the format described above is created, a script
 should exist which parses that file and generates 4 binaries from it, one for each EEPROM.

\section{Assembler}
Given a microcode generation tool, while it is now easy to write an \emph{instruction set} for the
16-bit breadboard computer, one would still have to write \emph{programs} in binary format. The
first level of abstraction away from writing binaries by hand should be an assembler.
The Assembler should fulfill the following requirements:
\begin{enumerate}
  \item  Dynamically recognize mnemonics based on a provided instruction definiton file
  \item  Recognise labels and replace them with memory locations
  \item  Recognise aliases and replace them accordingly
  \item  Recognise operands in decimal, hexadecimal and binary format
  \item Generate binaries based on an assembly source file and an instruction definiton file
\end{enumerate}

\section{Compiler}
To have a thorough understanding of the requirements and specifications for the \emph{WHILE}
language compiler, the language must be first presented and discussed.

\subsection{WHILE language}
The \emph{WHILE} language is a very simple Turing-complete programming language which is in syntax
very similar to pseudocode \ref{while-demo}.

\lstinputlisting[caption= \emph{WHILE} language Fibonacci program]{code/fib.while} \label{while-demo}

Some of the most important features of the \emph{WHILE} language are:
\begin{itemize}
  \item \emph{Limited types: } the language only has two types \emph{strings}, which can only be
  printed out, and \emph{integers}
  \item \emph{Simple statements: } the statements possible in the while language are \emph{while,
  if, assign, read, write}
  \item \emph{Restricted boolean and arithmetic expressions: } The while language an process data
  by executing instructions using simple arithmetic and boolean expressions. The allowed arithemtic
  operators are \emph{$+, -, *, /, \%$}. The allowed boolean operators are \emph{$\&\&, ||,  \leq, <, ==,
  >,  \geq$}
  \item \emph{Simplified I/O: } The \emph{WHILE} language uses the read and write statements to
  abstract the process of reading and writing data.
\end{itemize}

\subsection{Requirements}
Given the feature list of the \emph{WHILE} programming language described in the previous
section, a compiler should fulfill the following requirements:

\begin{itemize}
  \item Parse \emph{WHILE} source files
  \item Understand the underlying structure of the program
  \item Generate assembly code for a certain instruction set for the 16-bit breadboard computer
  which is built into the compiler.
\end{itemize}

\section{Programming Arduino driver}
In order to program the 16-bit breadboard computer with a binary generated by the assembler,
an Arduino driver coupled with a script to run on the main computer. The two pieces of software
should work in tandem to transmit a source file to the arduino and then on to the breadboard
computer. This should work as follows.

\subsection{Protocol specification}
\begin{enumerate}
  \item The Arduino is connected to the computer over USB
  \item The Arduino is manually reset
  \item The Arduino maintains an internal address counter, reset to 0
  \item The computer-side script reads a binary file and stores it in memory
  \item The computer sends the arduino two bytes at a time (two bytes form 1 breadboard
  computer instruction)
  \item The Arduino receives the 2 bytes and constructs a data word from them
  \item The Arduino programs the byte in the breadboard PC RAM at the address specified by its
  counter and then increments the counter
  \item The Arduino replies to the computer with a success message
  \item After the computer receives the success message, repeat from step 5
\end{enumerate}

\section{I/O Driver}
The I/O driver should work in a similar manner to the programming driver. It should allow the
the breadboard computer to interface with a personal computer by using the Arduino as an
information relay.

\subsection{Protocol Specification}
\begin{itemize}
  \item The Arduino is connected to the computer over USB
  \item The Arduino is manually reset
  \item The Arduino maintains an internal circular queue of 10 16-bit data words, which is
  initially reset to 0.
  \item If the computer sends data over to the Arduino, it should store that data at the end of
  the queue and update its queue pointer.
  \item If the breadboard computer triggers a read request, the Arduino should serve it a data
  word from the queue and then update its queue pointer.
  \item If the breadboard computer triggers a write request, the Arduino should read the data word
  found in the I/O module and then send it on to the main computer.
  \item The main computer should then display that data on the screen.
  \item At the start of the script on the main computer, a file should be provided which
  contains 10 numbers in either decimal, hexadecimal or binary format. This numbers should
  form the basis for the queue to be sent to the Arduino
\end{itemize}
