\chapter{Software Testing}

\section{Testing Methodology}
Since each piece of software was dependant on the software before it, each new tool
developed was tested during and immediatley after its development process. Taken
the relativley reduced complexity and size of each of the tools created, the decision
was made to skip \emph{unit testing} and only conduct \emph{integration testing}.

\section{Microcode programmer}
The microcode programmer was tested by writing an instrution set consisting of only one instruction
in both Python and YAML format and then generating binaries from that source. Then, the resulting
binaries were checked by hand to ensure that the right byte sequences were generated. After the
creation of richer instruction set, random samples were selected from the instruction set source
and then the binaries were checked for those particular instructions. \\
The microcode generator passed all tests successfully.

\section{Assembler and Compiler}

\paragraph{Assembler}
The Assembler was tested with multiple source files designed to incorporate all the features it
had specified: mnemonics, operators, aliases, labels. All files were assembled down to binaries
correctly.

\paragraph{Compiler}
Similarly, the compiler was checked against three WHILE language source files, two of which were used
as test files in the \textbf{6CCS3CFL} course thaught by \emph{Dr. Christian Urban} \cite{6ccs3cfl}.
All files were correctly compiled down to assembly.

\section{Arduino drivers}
\paragraph{Programming driver}
The programming driver was tested by generating zero-byte binaries, one-byte bineries and multiple
binaries with repeating patterns of bytes and then using the computer script and the driver software
to program the 16-bit breadboard computer. Then, using the manual programming mode to set the address
and tying the \emph{RO} (RAM-Out) signal high, random samples accross the entire memory space of the
computer were checked. All samples matched the expected bytes.

\paragraph{I/O driver}
The I/O driver was checked by having loading in sequences of \emph{LAI - IRA} instructions (Load A
immediate and I/O read A) with specific bit sequences as the operand for the \emph{LAI} instructions,
and then checking if the correct output was displayed back on the computer. To test the I/O reads,
multiple\emph{IWA} instructions (I/O write A) were executed together with specific queue values. It
was then checked if the expected value had been read into the A register. Unfortunatly,
at the time of writing, the I/O Arduino behaves iratically, only sometimes writing and reading from the
I/O module. It is unclear whether this is due to the software, the hardware or a combination of both.
This will be discussed as a point of imporvement in a following section.



\chapter{Evaluation}
To evaluate the success of the 16-bit breadboard computer architecture project, a parallel between
the goals set out at the start of the report and what was achieved in practice has to be drawn and
each success criterion has to be considered.

\section{Achieved Objectives}

\subsection{Analysis objectives}
\begin{enumerate}
  \item \emph{Computer architecture analysis:} this goal was successfully met in the background
  \ref{background} section of the report, where computer architecture, processor design,
  microarchitecture and processor component design were presented and discussed
  \item \emph{Analysis of the computer designed and built by Ben Eater: } \cite{eater2019breadboard}
  this objective was also successfully achieved in the background section of the report
  \ref{background}.
\end{enumerate}

\subsection{Hardware objectives}
\begin{enumerate}
  \item \emph{16 bit word length} : achieved both in design and implementation
  \item \emph{appropiatley sized memory space} : achieved both in design and implementation
  \item \emph{memory read/write capability} : achieved both in design and implementation
  \item \emph{capability to decode and execute instructions sequentially} : achieved both in design and implementation
  \item \emph{program counter alteration (jumps)} : achieved both in design and implementation
  \item \emph{I/O functionality} : This goal has not been fully achieved at the time of writing
  \item \emph{hardware-implemented ability to perform basic arithmetical operations} : achieved both in design and implementation
  \item \emph{simple branching} : achieved both in design and implementation
  \item \emph{variable speed clock} : achieved both in design and implementation
  \item \emph{single step clock function} achieved both in design and implementation
\end{enumerate}

\subsection{Software objectives}
\begin{enumerate}
  \item \emph{microcode generation tool} - completed successfully
  \item \emph{adequate microcode for the control mechanisms} - completed successfully
  \item \emph{assembly mnemonics} - completed successfully
  \item \emph{assembler package to turn assembly files into binaries} - completed successfully
  \item \emph{compiler for the WHILE-language to breadboard computer binaries} - partially completed.
  Discussed in the \emph{Unmet Objective} section \ref{unmet}
  \item \emph{Arduino driver software} - partially completed.
  Discussed in the \emph{Unmet Objective} section \ref{unmet}
\end{enumerate}

\section{Unmet Objectives} \label{unmet}
Both the hardware and software produced meet most of their set out objectives. That being said, a few
objectives have not been achieved fully. These will be discussed in this section of the report.

\subsection{I/O Issues}
While both the software and the hardware for the I/O module has been designed and built, it fails
to operate consistently. This makes deugging challenging, since there is no clear point of failure.
That being said, two theories have the largest likelyhood of successfully identifying the root cause.

\paragraph{Voltge spikes produced by the EEPROMs} After the incidient concerning the undefined
behaviour of the EEPROMs during an address change, this possibility first came to mind as a potential
source for the issues. The patch which was tried immediately was the insertion of RC-network filters
on the control signals which command the I/O module and the A register. This did not solve the abnormal
behaviour, but only filtering either one or the other module at one time was tried, not both at the
same time.

\paragraph{Code timing issues} It is possible that due to the fact that all events happend in the
computer at the same time, namely on the rising edge of the clock pulse, the interrupt triggered
on the arduino immediately reads the bus contents while they are still switching. This theory
is hard to prove though, since ISR (Interrupt Service Routine) of the Arduino has to be as short and
without delay calls.

\subsection{Compiler}
While the \emph{WHILE} language compiler works as specified and produced valid assembly files,
two features of the \emph{WHILE} language are missing from the implementation.

\paragraph{Strings} The \emph{WHILE} language has the option to print out strings of characters
to the screen. This feature has not been implemented in the Compiler for the language. That being
said, strings are not a part of the objectives set out for the hardware implementation, so there is
nothing to compile to. As such, it is debatable whether this constiutes an unmet objective.

\paragraph{Operators} Two arithmetical operators are missing from the \emph{WHILE} language compiler,
\emph{/ (Division)} and \emph{\% (the modulo operator)}.

\section{General Objective}
Given the main goal to create a hardware architecture which is both practically implementable
and easy to understand to a low level, as well as the creation of a suite of software tools to
accompany it and based on the list of achieved objectives, it is safe to say that the general objective
has been successfully achieved. The 16-bit breadboard computer is a physically implemented, it is
easy to understand how it functions and it can act as a great taching and learning medium for
computer architecture, design, implementation, general low-level programming as well as compilers and
assemblers.
